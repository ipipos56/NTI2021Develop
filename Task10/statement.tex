\assignementTitle{Перемещение в смежный сектор}{-}

Два автоматизированных погрузчика были активированы в разных помещениях логистического центра. 
У каждого робота в постоянной памяти есть карта всего логистического 
центра и информация о местоположении обоих автоматизированных погрузчиков. 

Помещения представляют из себя квадратные секции. 
За одну «фазу перемещения» погрузчик переходит прямо из одного помещения в другое, 
либо выполняет поворот на $90$ градусов внутри текущего помещения.

Первый погрузчик передвигается по «правилу левой руки», то есть движется вдоль левой стенки, 
когда спереди есть свободное помещение. Если слева стенки нет, то погрузчик должен повернуть 
налево на $90$ градусов и проехать одно помещение вперед. Если стенка есть слева и спереди, то погрузчик 
поворачивает направо на $90$ градусов.

Второй погрузчик может передвигаться в любом направлении. 
Его основная цель – в любой момент времени оказаться в смежном секторе с первым погрузчиком. 
Смежным сектором является сектор, находящийся рядом с первым погрузчиком в любой момент времени.

Первый погрузчик начинает двигаться из одного помещения в другое по «правилу левой руки». 
Необходимо определить через сколько «фаз перемещений» второй погрузчик окажется в смежном секторе с первым погрузчиком. 
Количество «фаз» должно быть минимально возможным.

Отсчет координат в логистическом центре начинается с левого верхнего угла. Ось Y направлена вниз, Ось X направлена вправо.

Направление положения погрузчика задаётся одной из четырёх букв (U, R, D, L). U - вверх, R - вправо, D - вниз, L - влево.


\inputfmtSection
Первая строка входных данных содержит 2 целых числа $y_{1}$, $x_{1}$ и одну заглавную букву $dir_{1}$,~ $dir_{1} \in {U,R,D,L}$ через пробел, где:
\begin{itemize}
    \item $y_{1}$ --- координата первого погрузчика по оси Y ($1 \leq y_{1} \leq 12$);
    \item $x_{1}$ --- координата первого погрузчика по оси X ($1 \leq x_{1} \leq 12$);
	\item $dir_{1}$ --- направление первого погрузчика ($1 \leq $dir_{1}$ \leq 12$);
\end{itemize}

Вторая строка входных данных содержит 2 целых числа $y_{2}$, $x_{2}$ и одну заглавную букву $dir_{2}$,~ $dir_{2} \in {U,R,D,L}$ через пробел, где:
\begin{itemize}
    \item $y_{2}$ --- координата второго погрузчика по оси Y ($1 \leq y_{2} \leq 12$);
    \item $x_{2}$ --- координата второго погрузчика по оси X ($1 \leq x_{2} \leq 12$);
	\item $dir_{2}$ --- направление второго погрузчика ($1 \leq $dir_{2}$ \leq 12$);
\end{itemize}

\outputfmtSection

Выведите единственное число – время на достижение вторым погрузчиком смежной клетки с первым (количество «фаз перемещений»).


\exampleSection

\sampleTitle{1}


\begin{myverbbox}[\small]{\vinput}
    7 6 D
	1 8 D
\end{myverbbox}
\begin{myverbbox}[\small]{\voutput}
    15
\end{myverbbox}
\inputoutputTable

\sampleTitle{2}

\begin{myverbbox}[\small]{\vinput}
    1 7 D
	7 1 D
\end{myverbbox}
\begin{myverbbox}[\small]{\voutput}
    31
\end{myverbbox}
\inputoutputTable


\includeSolutionIfExistsByPath{\SecondTourProgrDir}{10_task/solution}
